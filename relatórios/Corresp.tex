% Options for packages loaded elsewhere
\PassOptionsToPackage{unicode}{hyperref}
\PassOptionsToPackage{hyphens}{url}
%
\documentclass[
]{article}
\usepackage{lmodern}
\usepackage{amssymb,amsmath}
\usepackage{ifxetex,ifluatex}
\ifnum 0\ifxetex 1\fi\ifluatex 1\fi=0 % if pdftex
  \usepackage[T1]{fontenc}
  \usepackage[utf8]{inputenc}
  \usepackage{textcomp} % provide euro and other symbols
\else % if luatex or xetex
  \usepackage{unicode-math}
  \defaultfontfeatures{Scale=MatchLowercase}
  \defaultfontfeatures[\rmfamily]{Ligatures=TeX,Scale=1}
\fi
% Use upquote if available, for straight quotes in verbatim environments
\IfFileExists{upquote.sty}{\usepackage{upquote}}{}
\IfFileExists{microtype.sty}{% use microtype if available
  \usepackage[]{microtype}
  \UseMicrotypeSet[protrusion]{basicmath} % disable protrusion for tt fonts
}{}
\makeatletter
\@ifundefined{KOMAClassName}{% if non-KOMA class
  \IfFileExists{parskip.sty}{%
    \usepackage{parskip}
  }{% else
    \setlength{\parindent}{0pt}
    \setlength{\parskip}{6pt plus 2pt minus 1pt}}
}{% if KOMA class
  \KOMAoptions{parskip=half}}
\makeatother
\usepackage{xcolor}
\IfFileExists{xurl.sty}{\usepackage{xurl}}{} % add URL line breaks if available
\IfFileExists{bookmark.sty}{\usepackage{bookmark}}{\usepackage{hyperref}}
\hypersetup{
  pdftitle={Indicador de correspondência},
  hidelinks,
  pdfcreator={LaTeX via pandoc}}
\urlstyle{same} % disable monospaced font for URLs
\usepackage[margin=1in]{geometry}
\usepackage{color}
\usepackage{fancyvrb}
\newcommand{\VerbBar}{|}
\newcommand{\VERB}{\Verb[commandchars=\\\{\}]}
\DefineVerbatimEnvironment{Highlighting}{Verbatim}{commandchars=\\\{\}}
% Add ',fontsize=\small' for more characters per line
\usepackage{framed}
\definecolor{shadecolor}{RGB}{248,248,248}
\newenvironment{Shaded}{\begin{snugshade}}{\end{snugshade}}
\newcommand{\AlertTok}[1]{\textcolor[rgb]{0.94,0.16,0.16}{#1}}
\newcommand{\AnnotationTok}[1]{\textcolor[rgb]{0.56,0.35,0.01}{\textbf{\textit{#1}}}}
\newcommand{\AttributeTok}[1]{\textcolor[rgb]{0.77,0.63,0.00}{#1}}
\newcommand{\BaseNTok}[1]{\textcolor[rgb]{0.00,0.00,0.81}{#1}}
\newcommand{\BuiltInTok}[1]{#1}
\newcommand{\CharTok}[1]{\textcolor[rgb]{0.31,0.60,0.02}{#1}}
\newcommand{\CommentTok}[1]{\textcolor[rgb]{0.56,0.35,0.01}{\textit{#1}}}
\newcommand{\CommentVarTok}[1]{\textcolor[rgb]{0.56,0.35,0.01}{\textbf{\textit{#1}}}}
\newcommand{\ConstantTok}[1]{\textcolor[rgb]{0.00,0.00,0.00}{#1}}
\newcommand{\ControlFlowTok}[1]{\textcolor[rgb]{0.13,0.29,0.53}{\textbf{#1}}}
\newcommand{\DataTypeTok}[1]{\textcolor[rgb]{0.13,0.29,0.53}{#1}}
\newcommand{\DecValTok}[1]{\textcolor[rgb]{0.00,0.00,0.81}{#1}}
\newcommand{\DocumentationTok}[1]{\textcolor[rgb]{0.56,0.35,0.01}{\textbf{\textit{#1}}}}
\newcommand{\ErrorTok}[1]{\textcolor[rgb]{0.64,0.00,0.00}{\textbf{#1}}}
\newcommand{\ExtensionTok}[1]{#1}
\newcommand{\FloatTok}[1]{\textcolor[rgb]{0.00,0.00,0.81}{#1}}
\newcommand{\FunctionTok}[1]{\textcolor[rgb]{0.00,0.00,0.00}{#1}}
\newcommand{\ImportTok}[1]{#1}
\newcommand{\InformationTok}[1]{\textcolor[rgb]{0.56,0.35,0.01}{\textbf{\textit{#1}}}}
\newcommand{\KeywordTok}[1]{\textcolor[rgb]{0.13,0.29,0.53}{\textbf{#1}}}
\newcommand{\NormalTok}[1]{#1}
\newcommand{\OperatorTok}[1]{\textcolor[rgb]{0.81,0.36,0.00}{\textbf{#1}}}
\newcommand{\OtherTok}[1]{\textcolor[rgb]{0.56,0.35,0.01}{#1}}
\newcommand{\PreprocessorTok}[1]{\textcolor[rgb]{0.56,0.35,0.01}{\textit{#1}}}
\newcommand{\RegionMarkerTok}[1]{#1}
\newcommand{\SpecialCharTok}[1]{\textcolor[rgb]{0.00,0.00,0.00}{#1}}
\newcommand{\SpecialStringTok}[1]{\textcolor[rgb]{0.31,0.60,0.02}{#1}}
\newcommand{\StringTok}[1]{\textcolor[rgb]{0.31,0.60,0.02}{#1}}
\newcommand{\VariableTok}[1]{\textcolor[rgb]{0.00,0.00,0.00}{#1}}
\newcommand{\VerbatimStringTok}[1]{\textcolor[rgb]{0.31,0.60,0.02}{#1}}
\newcommand{\WarningTok}[1]{\textcolor[rgb]{0.56,0.35,0.01}{\textbf{\textit{#1}}}}
\usepackage{graphicx,grffile}
\makeatletter
\def\maxwidth{\ifdim\Gin@nat@width>\linewidth\linewidth\else\Gin@nat@width\fi}
\def\maxheight{\ifdim\Gin@nat@height>\textheight\textheight\else\Gin@nat@height\fi}
\makeatother
% Scale images if necessary, so that they will not overflow the page
% margins by default, and it is still possible to overwrite the defaults
% using explicit options in \includegraphics[width, height, ...]{}
\setkeys{Gin}{width=\maxwidth,height=\maxheight,keepaspectratio}
% Set default figure placement to htbp
\makeatletter
\def\fps@figure{htbp}
\makeatother
\setlength{\emergencystretch}{3em} % prevent overfull lines
\providecommand{\tightlist}{%
  \setlength{\itemsep}{0pt}\setlength{\parskip}{0pt}}
\setcounter{secnumdepth}{-\maxdimen} % remove section numbering

\title{Indicador de correspondência}
\author{}
\date{\vspace{-2.5em}}

\begin{document}
\maketitle

\hypertarget{apresentauxe7uxe3o}{%
\subsection{Apresentação}\label{apresentauxe7uxe3o}}

Este relatório apresenta resultados preliminares da amálise exploratória
dos dados do Sistema de Gestão da Defesa Civil. Propõe-se a criação de
um novo indicador que pode ser usado para avaliar a confiabilidade dos
dados estudados bem como a eficiência dos canais de atendimento em triar
as solicitações que entram no sistema.

O indicador consiste em um número em (\%), representando a fração das
amastras em que o ocorrência indicada na solicitação (ocorr\_solic) é
igual à ocorrência observada em campo e registrada pelo técnico no
momento da vistoria (ocorr\_vist).

\hypertarget{os-dados}{%
\subsection{Os dados}\label{os-dados}}

Foram utilizados os regisstros do SGDC correspondentes ao período de
01/07/2004 a 31/12/2019, inicialmente contendo \textbf{147512
observações e 16 variáveis}.

\begin{Shaded}
\begin{Highlighting}[]
\KeywordTok{head}\NormalTok{(pdata)}
\end{Highlighting}
\end{Shaded}

\begin{verbatim}
##   cod_proc          data_solic cod_logra                 desc_logra
## 1   100990 2019-12-31 20:08:38      9330        RUA ÁLVARO BAQUEIRO
## 2   100989 2019-12-31 19:13:20      8116 ESTRADA DO COQUEIRO GRANDE
## 3    99790 2019-12-31 17:05:54      9126             RUA DA ADUTORA
## 4   100987 2019-12-31 16:49:36     21133     2ª TRAVESSA BOM PASTOR
## 5   100986 2019-12-31 16:36:40     23359           RUA OSCAR SEIXAS
## 6   100985 2019-12-31 16:32:40      9126             RUA DA ADUTORA
##   num_imov_solic        nome_bairro               desc_pb coord_xi_logra
## 1             6E             ITAPUA       ITAPUÃ/IPITANGA   -38.36292050
## 2      7-1ºANDAR  FAZENDA GRANDE II            CAJAZEIRAS   -38.39949100
## 3       121-CASA      SÃO CRISTOVÃO       ITAPUÃ/IPITANGA   -38.35523480
## 4        24-CASA CAMPINAS DE PIRAJÁ LIBERDADE/SAO CAETANO   -38.47318120
## 5            S/N             PIRAJA               VALÉRIA   -38.46944190
## 6         7-CASA      SAO CRISTOVAO       ITAPUÃ/IPITANGA   -38.35523480
##   coord_yi_logra           ocorr_solic desc_orig_solic tipo_vist_solic
## 1   -12.94140110  ALAGAMENTO DE IMÓVEL       199 / 156          IMÓVEL
## 2   -12.90519790           INFILTRAÇÃO       199 / 156          IMÓVEL
## 3   -12.91639220  ALAGAMENTO DE IMÓVEL       199 / 156          IMÓVEL
## 4   -12.92414910  ALAGAMENTO DE IMÓVEL       199 / 156          IMÓVEL
## 5   -12.90376450 AMEACA DE DESABAMENTO       199 / 156          IMÓVEL
## 6   -12.91639220  ALAGAMENTO DE IMÓVEL       199 / 156          IMÓVEL
##              ocorr_vist           data_vist tipo_vist_vist          desc_status
## 1  ALAGAMENTO DE IMÓVEL 2020-01-02 12:32:53         IMÓVEL            CONCLUÍDO
## 2     AVALIACAO DA AREA 2020-01-02 00:00:00         IMÓVEL          ENC. SOCIAL
## 3  ALAGAMENTO DE IMÓVEL 2020-01-02 00:00:00         IMÓVEL            CONCLUÍDO
## 4  ALAGAMENTO DE IMÓVEL 2020-01-03 09:41:21         IMÓVEL            CONCLUÍDO
## 5 AMEACA DE DESABAMENTO 2020-01-06 00:00:00         IMÓVEL            CONCLUÍDO
## 6  ALAGAMENTO DE IMÓVEL 2020-01-02 00:00:00         IMÓVEL ENC. SOCIAL E CHEFIA
\end{verbatim}

\hypertarget{filtragem}{%
\subsubsection{Filtragem}\label{filtragem}}

Para as análises pretendidas, o primeiro requisito é que fossem
selecionadas apenas as solicitações que foram atendidas e que possuem um
tipo de ocorrência registrado pelo técnico vistoriados. Assim, foram
eliminadas as observações de acordo com o valor das variáveis:

Status da solicitação (desc\_status): CANCELADO, BLOQUEADO, PENDENTE,
NÃO VISUALIZADA e VISTORIA PROGRAMADA. Canal de abertura (orig\_solic):
ABERTA EM CAMPO, OFÍCIO e OUVIDORIA.

Optou-se por suprimir as solicitações com orig\_solic = \emph{ABERTA EM
CAMPO} pois, como as solicitações são também abertas no momento da
vistoria pelo técnico vistoriador, ocorr\_solic e ocorr\_vist devem, por
regra, serem iguais, o que desbalancearia a amostra. As solicitações
abertas via \emph{OFÍCIO} e \emph{OUVIDORIA} foram desconsideradas por
fugirem do escopo de interesse da análise.

\begin{Shaded}
\begin{Highlighting}[]
\NormalTok{pdata <-}\StringTok{ }\NormalTok{pdata[}\OperatorTok{!}\NormalTok{(pdata}\OperatorTok{$}\NormalTok{desc_status }\OperatorTok{==}\StringTok{ "CANCELADO"}\NormalTok{),]}
\NormalTok{pdata <-}\StringTok{ }\NormalTok{pdata[}\OperatorTok{!}\NormalTok{(pdata}\OperatorTok{$}\NormalTok{desc_status }\OperatorTok{==}\StringTok{ "BLOQUEADO"}\NormalTok{),]}
\NormalTok{pdata <-}\StringTok{ }\NormalTok{pdata[}\OperatorTok{!}\NormalTok{(pdata}\OperatorTok{$}\NormalTok{desc_status }\OperatorTok{==}\StringTok{ "PENDENTE"}\NormalTok{),]}
\NormalTok{pdata <-}\StringTok{ }\NormalTok{pdata[}\OperatorTok{!}\NormalTok{(pdata}\OperatorTok{$}\NormalTok{desc_status }\OperatorTok{==}\StringTok{ "NÃO VISUALIZADA"}\NormalTok{),]}
\NormalTok{pdata <-}\StringTok{ }\NormalTok{pdata[}\OperatorTok{!}\NormalTok{(pdata}\OperatorTok{$}\NormalTok{desc_status }\OperatorTok{==}\StringTok{ "VISTORIA PROGRAMADA"}\NormalTok{),]}
\NormalTok{pdata <-}\StringTok{ }\NormalTok{pdata[}\OperatorTok{!}\NormalTok{(pdata}\OperatorTok{$}\NormalTok{desc_orig_solic }\OperatorTok{==}\StringTok{ "ABERTA EM CAMPO"}\NormalTok{),]}
\NormalTok{pdata <-}\StringTok{ }\NormalTok{pdata[}\OperatorTok{!}\NormalTok{(pdata}\OperatorTok{$}\NormalTok{desc_orig_solic }\OperatorTok{==}\StringTok{ "OFICIO"}\NormalTok{),]}
\NormalTok{pdata <-}\StringTok{ }\NormalTok{pdata[}\OperatorTok{!}\NormalTok{(pdata}\OperatorTok{$}\NormalTok{desc_orig_solic }\OperatorTok{==}\StringTok{ "OUVIDORIA"}\NormalTok{),]}
\NormalTok{pdata <-}\StringTok{ }\NormalTok{pdata[}\OperatorTok{!}\NormalTok{(pdata}\OperatorTok{$}\NormalTok{ocorr_vist }\OperatorTok{==}\StringTok{ "NULL"}\NormalTok{),]}
\end{Highlighting}
\end{Shaded}

Também para facilitar a análise, as ocorrências \emph{AVLIAÇÃO DE IMÓVEL
ALAGADO}, \emph{ALAGAMENTO DE IMÓVEL} e \emph{ALAGAMENTO DE IMÓVE} foram
sintetizadas em um único tipo \emph{ALAGAMENTO}.

\hypertarget{modelagem}{%
\subsubsection{Modelagem}\label{modelagem}}

\begin{Shaded}
\begin{Highlighting}[]
\NormalTok{alagamentos <-}\StringTok{ }\KeywordTok{c}\NormalTok{(}\StringTok{"AVALIAÇÃO DE IMÓVEL ALAGADO"}\NormalTok{, }\StringTok{"ALAGAMENTO DE IMÓVEL"}\NormalTok{, }\StringTok{"ALAGAMENTO DE ÁREA"}\NormalTok{)}
\NormalTok{pdata}\OperatorTok{$}\NormalTok{ocorr_solic <-}\StringTok{ }\KeywordTok{ifelse}\NormalTok{(pdata}\OperatorTok{$}\NormalTok{ocorr_solic }\OperatorTok\StringTok{ }\NormalTok{alagamentos, }\StringTok{"ALAGAMENTO"}\NormalTok{, pdata}\OperatorTok{$}\NormalTok{ocorr_solic)}
\NormalTok{pdata}\OperatorTok{$}\NormalTok{ocorr_vist <-}\StringTok{ }\KeywordTok{ifelse}\NormalTok{(pdata}\OperatorTok{$}\NormalTok{ocorr_vist }\OperatorTok\StringTok{ }\NormalTok{alagamentos, }\StringTok{"ALAGAMENTO"}\NormalTok{, pdata}\OperatorTok{$}\NormalTok{ocorr_vist)}
\end{Highlighting}
\end{Shaded}

Por fim, criamos uma nova coluna no dataset, que deverá ter valor
VERDADEIRO ( \emph{TRUE} ) se a ocorrência indicada na solicitação e a
observada na vistoria coincidirem e FALSO ( \emph{FALSE} ) caso sejam
diferentes.

O conjunto de dados final contem 90.046 observações e 17 variáveis.

\begin{verbatim}
##   cod_proc          data_solic cod_logra                 desc_logra
## 1   100990 2019-12-31 20:08:38      9330        RUA ÁLVARO BAQUEIRO
## 2   100989 2019-12-31 19:13:20      8116 ESTRADA DO COQUEIRO GRANDE
## 3    99790 2019-12-31 17:05:54      9126             RUA DA ADUTORA
## 4   100987 2019-12-31 16:49:36     21133     2ª TRAVESSA BOM PASTOR
## 5   100986 2019-12-31 16:36:40     23359           RUA OSCAR SEIXAS
## 6   100985 2019-12-31 16:32:40      9126             RUA DA ADUTORA
##   num_imov_solic        nome_bairro               desc_pb coord_xi_logra
## 1             6E             ITAPUA       ITAPUÃ/IPITANGA   -38.36292050
## 2      7-1ºANDAR  FAZENDA GRANDE II            CAJAZEIRAS   -38.39949100
## 3       121-CASA      SÃO CRISTOVÃO       ITAPUÃ/IPITANGA   -38.35523480
## 4        24-CASA CAMPINAS DE PIRAJÁ LIBERDADE/SAO CAETANO   -38.47318120
## 5            S/N             PIRAJA               VALÉRIA   -38.46944190
## 6         7-CASA      SAO CRISTOVAO       ITAPUÃ/IPITANGA   -38.35523480
##   coord_yi_logra           ocorr_solic desc_orig_solic tipo_vist_solic
## 1   -12.94140110            ALAGAMENTO       199 / 156          IMÓVEL
## 2   -12.90519790           INFILTRAÇÃO       199 / 156          IMÓVEL
## 3   -12.91639220            ALAGAMENTO       199 / 156          IMÓVEL
## 4   -12.92414910            ALAGAMENTO       199 / 156          IMÓVEL
## 5   -12.90376450 AMEACA DE DESABAMENTO       199 / 156          IMÓVEL
## 6   -12.91639220            ALAGAMENTO       199 / 156          IMÓVEL
##              ocorr_vist           data_vist tipo_vist_vist          desc_status
## 1            ALAGAMENTO 2020-01-02 12:32:53         IMÓVEL            CONCLUÍDO
## 2     AVALIACAO DA AREA 2020-01-02 00:00:00         IMÓVEL          ENC. SOCIAL
## 3            ALAGAMENTO 2020-01-02 00:00:00         IMÓVEL            CONCLUÍDO
## 4            ALAGAMENTO 2020-01-03 09:41:21         IMÓVEL            CONCLUÍDO
## 5 AMEACA DE DESABAMENTO 2020-01-06 00:00:00         IMÓVEL            CONCLUÍDO
## 6            ALAGAMENTO 2020-01-02 00:00:00         IMÓVEL ENC. SOCIAL E CHEFIA
##   match
## 1  TRUE
## 2 FALSE
## 3  TRUE
## 4  TRUE
## 5  TRUE
## 6  TRUE
\end{verbatim}

\hypertarget{anuxe1lises}{%
\subsection{Análises}\label{anuxe1lises}}

\hypertarget{univariada}{%
\subsubsection{Univariada}\label{univariada}}

O valor de C para cada ocorrência pode ser obsevado no gráfico a seguir.
Observa-se que quatro ocorrências críticas, que exigem uma resposta
imediata da Defesa Civil e demais órgãos do SMPDC, mostram valores entre
50\% e 60\%, enquanto outras três do tipo ficam abaixo dos 40\%. Podemos
dizer a partir daí que uma solicitação aberta como \textbf{deslizamento
de terra} tem uma chance aproximada de 50\% de \textbf{não} ser um
deslizamento de terra real, as mesmas chances de tirar um ``cara'' ao
lançar uma moeda.

\includegraphics{Corresp_files/figure-latex/unnamed-chunk-6-1.pdf} É
possível também, a partir desta primeira análise, investigar quais são
as situações reais observadas pelo técnico em campo quando, não são as
ocorrências indicadas no momento da abertura do chamado. Abaixo, temos a
tabela com essas informações para as ocorrências destacadas no gráfico
anterior.

\begin{Shaded}
\begin{Highlighting}[]
\NormalTok{prop_}\DecValTok{3}\NormalTok{ <-}\StringTok{ }\NormalTok{sdata[sdata}\OperatorTok{$}\NormalTok{ocorr_solic }\OperatorTok\StringTok{ }\NormalTok{ocorr_emerg,] }\OperatorTok\StringTok{ }\KeywordTok{group_by}\NormalTok{(ocorr_solic, ocorr_vist) }\OperatorTok\StringTok{ }
\StringTok{          }\KeywordTok{summarise}\NormalTok{(}\DataTypeTok{n =} \KeywordTok{n}\NormalTok{()) }\OperatorTok\StringTok{ }
\StringTok{          }\KeywordTok{mutate}\NormalTok{(}\DataTypeTok{freq =}\NormalTok{ n}\OperatorTok{/}\KeywordTok{sum}\NormalTok{(n)}\OperatorTok{*}\DecValTok{100}\NormalTok{) }\OperatorTok
\StringTok{          }\KeywordTok{arrange}\NormalTok{(}\OperatorTok{-}\NormalTok{freq, }\DataTypeTok{.by_group =} \OtherTok{TRUE}\NormalTok{) }\OperatorTok\StringTok{ }
\StringTok{          }\KeywordTok{mutate}\NormalTok{(}\DataTypeTok{ocorr_vist =} \KeywordTok{factor}\NormalTok{(ocorr_vist, }\DataTypeTok{levels =}\NormalTok{ ocorr_vist)) }\OperatorTok\StringTok{ }
\StringTok{          }\KeywordTok{select}\NormalTok{(}\OperatorTok{-}\KeywordTok{c}\NormalTok{(n)) }\OperatorTok\StringTok{ }
\StringTok{          }\KeywordTok{as.data.frame}\NormalTok{()}
\end{Highlighting}
\end{Shaded}

\begin{verbatim}
## `summarise()` regrouping output by 'ocorr_solic' (override with `.groups` argument)
\end{verbatim}

Calculando essa tabela para \emph{ORITENTAÇÃO TÉCNCIA}, um tipo de
ocorrência coringa dentro do sistema, selecionado quando a ocorrência
observada não se encaixa em nenhuma das tipologias disponíveis,
observamos que aproximadamente 1/4 das solicitações são indicadas
posteriormente pelo técnico como \emph{AMEAÇA DE DESABAMENTO}. A maior
parte é classidicada tanbém como \emph{ORIENTAÇÃO TÉCNICA}, o que abre
espaço para novas análises mais aprofundadas para verificar, talvez a
partir das intervenções indicadas, se são atendimentos que fogem do
escopo da Defesa Civil, ou mesmo identificar novas tipologias a serem
inseridas na lista de ocorrências.

\begin{Shaded}
\begin{Highlighting}[]
\NormalTok{sdata[sdata}\OperatorTok{$}\NormalTok{ocorr_solic }\OperatorTok{==}\StringTok{ "ORIENTAÇÃO TÉCNICA"}\NormalTok{,] }\OperatorTok\StringTok{ }\KeywordTok{group_by}\NormalTok{(ocorr_vist) }\OperatorTok\StringTok{ }
\StringTok{          }\KeywordTok{summarise}\NormalTok{(}\DataTypeTok{n =} \KeywordTok{n}\NormalTok{()) }\OperatorTok\StringTok{ }
\StringTok{          }\KeywordTok{mutate}\NormalTok{(}\DataTypeTok{freq =}\NormalTok{ n}\OperatorTok{/}\KeywordTok{sum}\NormalTok{(n)}\OperatorTok{*}\DecValTok{100}\NormalTok{) }\OperatorTok\StringTok{ }
\StringTok{          }\KeywordTok{arrange}\NormalTok{(}\OperatorTok{-}\NormalTok{freq)}
\end{Highlighting}
\end{Shaded}

\begin{verbatim}
## `summarise()` ungrouping output (override with `.groups` argument)
\end{verbatim}

\begin{verbatim}
## # A tibble: 19 x 3
##    ocorr_vist                               n    freq
##    <chr>                                <int>   <dbl>
##  1 ORIENTAÇÃO TÉCNICA                    1715 54.7   
##  2 AMEACA DE DESABAMENTO                  762 24.3   
##  3 AMEACA DE DESLIZAMENTO                 190  6.05  
##  4 INFILTRAÇÃO                            161  5.13  
##  5 ALAGAMENTO                             102  3.25  
##  6 DESLIZAMENTO DE TERRA                   35  1.12  
##  7 AVALIACAO DA AREA                       31  0.988 
##  8 DESABAMENTO DE IMOVEL                   29  0.924 
##  9 AMEACA DE DESABAMENTO DE MURO           26  0.829 
## 10 INCENDIO                                24  0.765 
## 11 DESABAMENTO PARCIAL                     15  0.478 
## 12 ARVORE AMEACANDO CAIR                   14  0.446 
## 13 DESABAMENTO DE MURO                     13  0.414 
## 14 PISTA ROMPIDA                           12  0.382 
## 15 DESTELHAMENTO                            4  0.127 
## 16 POSTE AMEAÇANDO CAIR                     2  0.0637
## 17 ARMAZENAMENTO DE MATERIAIS PERIGOSOS     1  0.0319
## 18 ARVORE CAIDA                             1  0.0319
## 19 GALHO DE ÁRVORE CAÍDO                    1  0.0319
\end{verbatim}

\hypertarget{multivariada}{%
\subsubsection{Multivariada}\label{multivariada}}

Nesta seção dispomos as observações feitas a partir do cruzamento do
Indicador (C) com outras variáveis, como canal de ocorrência, data de
abertura do chamado e local.

\hypertarget{qauntidade-de-solicitauxe7uxf5es-e-data-de-abertura}{%
\paragraph{Qauntidade de solicitações e data de
abertura}\label{qauntidade-de-solicitauxe7uxf5es-e-data-de-abertura}}

Para deslizamentos de terra, a variação de C reduz à medida que o número
de solicitações aumenta. Na prática, significa que em dias com muitos
chamados de deslizamentos de terra, aqui estudamos um número base de 100
solicitações, o valor de C estabilizia entre 50\% e 60\%.

\begin{Shaded}
\begin{Highlighting}[]
\NormalTok{os_dt =}\StringTok{ }\KeywordTok{c}\NormalTok{(}\StringTok{"DESLIZAMENTO DE TERRA"}\NormalTok{, }\StringTok{"AMEACA DE DESLIZAMENTO"}\NormalTok{)}
\NormalTok{os_db =}\StringTok{ }\KeywordTok{c}\NormalTok{(}\StringTok{"DESABAMENTO DE IMOVEL"}\NormalTok{, }\StringTok{"AMEACA DE DESABAMENTO"}\NormalTok{)}

\NormalTok{sdata[sdata}\OperatorTok{$}\NormalTok{ocorr_solic }\OperatorTok\StringTok{ }\NormalTok{os_dt,] }\OperatorTok\StringTok{ }\KeywordTok{group_by}\NormalTok{(}\KeywordTok{year}\NormalTok{(data_solic), }\KeywordTok{date}\NormalTok{(data_solic), ocorr_solic) }\OperatorTok\StringTok{ }
\StringTok{                                          }\KeywordTok{summarise}\NormalTok{(}\DataTypeTok{n_solic =} \KeywordTok{n}\NormalTok{(), }\DataTypeTok{C =} \KeywordTok{mean}\NormalTok{(match)) }\OperatorTok\StringTok{ }
\StringTok{                                          }\KeywordTok{filter}\NormalTok{(n_solic }\OperatorTok{>=}\StringTok{ }\DecValTok{0}\NormalTok{, }\StringTok{`}\DataTypeTok{year(data_solic)}\StringTok{`} \OperatorTok{>=}\StringTok{ }\DecValTok{2005}\NormalTok{, ocorr_solic}\OperatorTok{==}\StringTok{"DESLIZAMENTO DE TERRA"}\NormalTok{) }\OperatorTok\StringTok{ }
\StringTok{                                          }\KeywordTok{ggplot}\NormalTok{(}\KeywordTok{aes}\NormalTok{(}\DataTypeTok{x=}\NormalTok{n_solic, }\DataTypeTok{y=}\NormalTok{C, }\DataTypeTok{group =}\NormalTok{ ocorr_solic)) }\OperatorTok{+}\StringTok{ }\KeywordTok{geom_point}\NormalTok{(}\KeywordTok{aes}\NormalTok{(}\DataTypeTok{fill=}\NormalTok{ocorr_solic)) }\OperatorTok{+}\StringTok{ }\KeywordTok{coord_cartesian}\NormalTok{(}\DataTypeTok{ylim =} \KeywordTok{c}\NormalTok{(}\DecValTok{0}\NormalTok{,}\DecValTok{1}\NormalTok{))}\OperatorTok{+}
\StringTok{                                          }\KeywordTok{stat_smooth}\NormalTok{(}\DataTypeTok{method =} \StringTok{"lm"}\NormalTok{, }\DataTypeTok{col =} \StringTok{"red"}\NormalTok{)}\OperatorTok{+}\StringTok{ }\CommentTok{#facet_wrap(~ocorr_solic)+ }
\StringTok{                                          }\KeywordTok{labs}\NormalTok{(}\DataTypeTok{title =} \StringTok{"Gráfico de dispersão: número de solicitações (n_solic) x indicador (C)"}\NormalTok{,}
                                                \DataTypeTok{subtitle =} \StringTok{"Deslizamentos de terra. n de corte = 100"}\NormalTok{, }\DataTypeTok{x =} \StringTok{"número de solicitações"}\NormalTok{, }\DataTypeTok{y =} \StringTok{"indicador"}\NormalTok{)}
\end{Highlighting}
\end{Shaded}

\begin{verbatim}
## `summarise()` regrouping output by 'year(data_solic)', 'date(data_solic)' (override with `.groups` argument)
\end{verbatim}

\begin{verbatim}
## `geom_smooth()` using formula 'y ~ x'
\end{verbatim}

\includegraphics{Corresp_files/figure-latex/unnamed-chunk-9-1.pdf}

\begin{Shaded}
\begin{Highlighting}[]
\NormalTok{sdata[sdata}\OperatorTok{$}\NormalTok{ocorr_solic }\OperatorTok\StringTok{ }\NormalTok{os_dt,] }\OperatorTok\StringTok{ }\KeywordTok{group_by}\NormalTok{(}\KeywordTok{year}\NormalTok{(data_solic), }\KeywordTok{date}\NormalTok{(data_solic), ocorr_solic) }\OperatorTok\StringTok{ }
\StringTok{                                          }\KeywordTok{summarise}\NormalTok{(}\DataTypeTok{n_solic =} \KeywordTok{n}\NormalTok{(), }\DataTypeTok{C =} \KeywordTok{mean}\NormalTok{(match)) }\OperatorTok\StringTok{ }
\StringTok{                                          }\KeywordTok{filter}\NormalTok{(n_solic }\OperatorTok{>=}\StringTok{ }\DecValTok{10}\NormalTok{, }\StringTok{`}\DataTypeTok{year(data_solic)}\StringTok{`} \OperatorTok{>=}\StringTok{ }\DecValTok{2005}\NormalTok{) }\OperatorTok\StringTok{ }
\StringTok{                                          }\KeywordTok{ggplot}\NormalTok{() }\OperatorTok{+}\StringTok{ }\KeywordTok{geom_histogram}\NormalTok{(}\KeywordTok{aes}\NormalTok{(}\DataTypeTok{x=}\NormalTok{C))}
\end{Highlighting}
\end{Shaded}

\begin{verbatim}
## `summarise()` regrouping output by 'year(data_solic)', 'date(data_solic)' (override with `.groups` argument)
\end{verbatim}

\begin{verbatim}
## `stat_bin()` using `bins = 30`. Pick better value with `binwidth`.
\end{verbatim}

\includegraphics{Corresp_files/figure-latex/unnamed-chunk-9-2.pdf}

\begin{Shaded}
\begin{Highlighting}[]
\NormalTok{dt <-}\StringTok{ }\NormalTok{sdata[sdata}\OperatorTok{$}\NormalTok{ocorr_solic }\OperatorTok\StringTok{ }\NormalTok{os_dt,] }\OperatorTok\StringTok{ }\KeywordTok{group_by}\NormalTok{(}\KeywordTok{year}\NormalTok{(data_solic), }\KeywordTok{date}\NormalTok{(data_solic), ocorr_solic) }\OperatorTok\StringTok{ }
\StringTok{                                          }\KeywordTok{summarise}\NormalTok{(}\DataTypeTok{n_solic =} \KeywordTok{n}\NormalTok{(), }\DataTypeTok{C =} \KeywordTok{mean}\NormalTok{(match)) }\OperatorTok\StringTok{ }
\StringTok{                                          }\KeywordTok{filter}\NormalTok{(n_solic }\OperatorTok{>=}\StringTok{ }\DecValTok{100}\NormalTok{, }\StringTok{`}\DataTypeTok{year(data_solic)}\StringTok{`} \OperatorTok{>=}\StringTok{ }\DecValTok{2005}\NormalTok{, ocorr_solic }\OperatorTok{==}\StringTok{ "DESLIZAMENTO DE TERRA"}\NormalTok{) }
\end{Highlighting}
\end{Shaded}

\begin{verbatim}
## `summarise()` regrouping output by 'year(data_solic)', 'date(data_solic)' (override with `.groups` argument)
\end{verbatim}

\begin{Shaded}
\begin{Highlighting}[]
\KeywordTok{cor}\NormalTok{(dt}\OperatorTok{$}\NormalTok{n_solic, dt}\OperatorTok{$}\NormalTok{C,  }\DataTypeTok{method =} \KeywordTok{c}\NormalTok{(}\StringTok{"pearson"}\NormalTok{))}
\end{Highlighting}
\end{Shaded}

\begin{verbatim}
## [1] 0.490569
\end{verbatim}

\begin{Shaded}
\begin{Highlighting}[]
\NormalTok{sdata[sdata}\OperatorTok{$}\NormalTok{ocorr_solic }\OperatorTok\StringTok{ }\NormalTok{os_dt,] }\OperatorTok\StringTok{ }\KeywordTok{group_by}\NormalTok{(}\KeywordTok{year}\NormalTok{(data_solic), }\KeywordTok{date}\NormalTok{(data_solic), ocorr_solic) }\OperatorTok\StringTok{ }
\StringTok{                                          }\KeywordTok{summarise}\NormalTok{(}\DataTypeTok{n_solic =} \KeywordTok{n}\NormalTok{(), }\DataTypeTok{C =} \KeywordTok{mean}\NormalTok{(match)) }\OperatorTok\StringTok{ }
\StringTok{                                          }\KeywordTok{filter}\NormalTok{(n_solic }\OperatorTok{>=}\StringTok{ }\DecValTok{0}\NormalTok{, }\StringTok{`}\DataTypeTok{year(data_solic)}\StringTok{`} \OperatorTok{>=}\StringTok{ }\DecValTok{2005}\NormalTok{, ocorr_solic}\OperatorTok{==}\StringTok{"DESLIZAMENTO DE TERRA"}\NormalTok{) }\OperatorTok\StringTok{ }
\StringTok{                                          }\KeywordTok{ggplot}\NormalTok{(}\KeywordTok{aes}\NormalTok{(}\DataTypeTok{x=}\NormalTok{n_solic, }\DataTypeTok{y=}\NormalTok{C, }\DataTypeTok{group =}\NormalTok{ ocorr_solic)) }\OperatorTok{+}\StringTok{ }\KeywordTok{geom_point}\NormalTok{(}\KeywordTok{aes}\NormalTok{(}\DataTypeTok{fill=}\NormalTok{ocorr_solic)) }\OperatorTok{+}\StringTok{ }\KeywordTok{coord_cartesian}\NormalTok{(}\DataTypeTok{ylim =} \KeywordTok{c}\NormalTok{(}\DecValTok{0}\NormalTok{,}\DecValTok{1}\NormalTok{))}\OperatorTok{+}
\StringTok{                                          }\CommentTok{#stat_smooth(method = "lm", col = "red")+}
\StringTok{                                          }\CommentTok{#facet_wrap(~`year(data_solic)`)+ }
\StringTok{                                          }\KeywordTok{labs}\NormalTok{(}\DataTypeTok{title =} \StringTok{"Gráfico de dispersão: número de solicitações (n_solic) x indicador (C)"}\NormalTok{,}
                                                \DataTypeTok{subtitle =} \StringTok{"Deslizamentos de terra. n de corte = 100"}\NormalTok{, }\DataTypeTok{x =} \StringTok{"número de solicitações"}\NormalTok{, }\DataTypeTok{y =} \StringTok{"indicador"}\NormalTok{)}\OperatorTok{+}
\StringTok{                                          }\KeywordTok{transition_reveal}\NormalTok{(}\StringTok{`}\DataTypeTok{year(data_solic)}\StringTok{`}\NormalTok{)}
\end{Highlighting}
\end{Shaded}

\begin{verbatim}
## `summarise()` regrouping output by 'year(data_solic)', 'date(data_solic)' (override with `.groups` argument)
\end{verbatim}

\begin{verbatim}
## Warning: No renderer available. Please install the gifski, av, or magick package
## to create animated output
\end{verbatim}

\hypertarget{canal-de-abertura}{%
\paragraph{Canal de abertura}\label{canal-de-abertura}}

Estudar a variação de (C) de acordo com a forma como as solicitações
entram no sistema pode nos dar uma dimensão da qualidade da triagem
feita pelos canais de atendimento. Espera-se que o atendente seja capaz
de ouvir o relato do cidadão e registrá-lo no sistema de forma concisa,
captando os detalhes cruciais e fornecendo aos setores responsáveis pelo
roteiro de vistoria informções confiáveis para definir a prioridade de
atendimento pelo órgão.

Para as análises a seguir, serão consideradas as ocorrências de
\emph{AMEAÇA DE DESLIZAMENTO} e \emph{DESLIZAMENTO DE TERRA}, dada a sua
relevância na rotina do órgão e sua homogeneidade durante o periíodo
analisado.

Sob essa perspectiva, analisamos a primeira hipótese, que é um senso
comum dentro da CODESAL: \emph{``Desde que saiu da sede do órgão, em
2015, a qualidade do atendimento 199 caiu''}. Essa afirmação parte da
premissa que, enquanto na sede, os atendentes do 199 poderiam ser melhor
orientados pelos técnicos, podendo tirar dúvidas e recebendo correções
de imediato. Além disso, agora que estão trabalhando no formato de Call
Center, os atendentes recbem chamados não só do 199 mas também do
telefone da Ouvidoria Geral do Município. É rezoável considerar, então,
que o atendimento presencial da CODESAL, aqui entitulado
\emph{PESSOALMENTE} deve apresentar um valor de C no mínimo maior que o
valor para os atendimentos feitos pelo 199.

Contudo, o que observamos nos gráficos abaixo nega a hipótese: além de
não podermos dizer que houve uma queda de C para o 199 a partir de 2015,
para deslizamentos de terra o 199 é mais acertivo que o atendimento
presencial

\begin{Shaded}
\begin{Highlighting}[]
\NormalTok{sdata[sdata}\OperatorTok{$}\NormalTok{ocorr_solic }\OperatorTok\StringTok{ }\NormalTok{os_dt }\OperatorTok{&}\StringTok{ }\NormalTok{sdata}\OperatorTok{$}\NormalTok{desc_orig_solic }\OperatorTok\StringTok{ }\KeywordTok{c}\NormalTok{(}\StringTok{"199"}\NormalTok{, }\StringTok{"PESSOALMENTE"}\NormalTok{),] }\OperatorTok\StringTok{ }\KeywordTok{group_by}\NormalTok{(}\KeywordTok{year}\NormalTok{(data_solic), desc_orig_solic) }\OperatorTok\StringTok{ }
\StringTok{                                         }\KeywordTok{summarise}\NormalTok{(}\DataTypeTok{n =} \KeywordTok{mean}\NormalTok{(match)}\OperatorTok{*}\DecValTok{100}\NormalTok{) }\OperatorTok\StringTok{ }
\StringTok{                                         }\KeywordTok{ggplot}\NormalTok{() }\OperatorTok{+}\StringTok{ }\KeywordTok{geom_line}\NormalTok{(}\KeywordTok{aes}\NormalTok{(}\DataTypeTok{x =} \StringTok{`}\DataTypeTok{year(data_solic)}\StringTok{`}\NormalTok{, }\DataTypeTok{y =}\NormalTok{ n, }\DataTypeTok{colour =} \KeywordTok{as.factor}\NormalTok{(desc_orig_solic)), }\DataTypeTok{size =} \FloatTok{1.2}\NormalTok{) }\OperatorTok{+}\StringTok{                      }
\StringTok{                                          }\KeywordTok{coord_cartesian}\NormalTok{(}\DataTypeTok{xlim =}\KeywordTok{c}\NormalTok{(}\DecValTok{2005}\NormalTok{, }\DecValTok{2019}\NormalTok{), }\DataTypeTok{ylim =} \KeywordTok{c}\NormalTok{(}\DecValTok{0}\NormalTok{, }\DecValTok{100}\NormalTok{)) }\OperatorTok{+}\StringTok{   }
\StringTok{                                         }\KeywordTok{scale_x_continuous}\NormalTok{(}\DataTypeTok{breaks =} \KeywordTok{seq}\NormalTok{(}\DecValTok{2005}\NormalTok{,}\DecValTok{2019}\NormalTok{, }\DecValTok{2}\NormalTok{))}\OperatorTok{+}
\StringTok{                                         }\KeywordTok{theme}\NormalTok{(}\DataTypeTok{legend.position =} \StringTok{"bottom"}\NormalTok{, }\DataTypeTok{legend.box =} \StringTok{"horizontal"}\NormalTok{)}\OperatorTok{+}
\StringTok{                                          }\KeywordTok{labs}\NormalTok{(}\DataTypeTok{title =} \StringTok{"Indicador de verdadeiro-positivo (C) por canal de atendimento"}\NormalTok{,}
                                              \DataTypeTok{caption =} \StringTok{"Período analisado: 07/05/2020 a 31/12/2019"}\NormalTok{, }
                                              \DataTypeTok{x =} \StringTok{"Ano"}\NormalTok{,}
                                              \DataTypeTok{y =} \StringTok{"C"}\NormalTok{,}
                                              \DataTypeTok{colour =} \StringTok{"Canal de aberura"}\NormalTok{) }
\end{Highlighting}
\end{Shaded}

\begin{verbatim}
## `summarise()` regrouping output by 'year(data_solic)' (override with `.groups` argument)
\end{verbatim}

\includegraphics{Corresp_files/figure-latex/unnamed-chunk-10-1.pdf}

\begin{Shaded}
\begin{Highlighting}[]
\NormalTok{sdata[sdata}\OperatorTok{$}\NormalTok{ocorr_solic }\OperatorTok\StringTok{ }\NormalTok{os_dt }\OperatorTok{&}\StringTok{ }\NormalTok{sdata}\OperatorTok{$}\NormalTok{desc_orig_solic }\OperatorTok\StringTok{ }\KeywordTok{c}\NormalTok{(}\StringTok{"199"}\NormalTok{, }\StringTok{"PESSOALMENTE"}\NormalTok{),] }\OperatorTok\StringTok{ }\KeywordTok{group_by}\NormalTok{(}\KeywordTok{year}\NormalTok{(data_solic), ocorr_solic, desc_orig_solic) }\OperatorTok\StringTok{ }
\StringTok{                                         }\KeywordTok{summarise}\NormalTok{(}\DataTypeTok{n =} \KeywordTok{mean}\NormalTok{(match)}\OperatorTok{*}\DecValTok{100}\NormalTok{) }\OperatorTok\StringTok{ }
\StringTok{                                         }\KeywordTok{ggplot}\NormalTok{() }\OperatorTok{+}\StringTok{ }\KeywordTok{geom_line}\NormalTok{(}\KeywordTok{aes}\NormalTok{(}\DataTypeTok{x =} \StringTok{`}\DataTypeTok{year(data_solic)}\StringTok{`}\NormalTok{, }\DataTypeTok{y =}\NormalTok{ n, }\DataTypeTok{colour =} \KeywordTok{as.factor}\NormalTok{(ocorr_solic)), }\DataTypeTok{size =} \FloatTok{1.2}\NormalTok{) }\OperatorTok{+}\StringTok{                                                                 }\CommentTok{#facet_wrap(~desc_orig_solic) +}
\StringTok{                                          }\KeywordTok{coord_cartesian}\NormalTok{(}\DataTypeTok{xlim =}\KeywordTok{c}\NormalTok{(}\DecValTok{2005}\NormalTok{, }\DecValTok{2019}\NormalTok{), }\DataTypeTok{ylim =} \KeywordTok{c}\NormalTok{(}\DecValTok{0}\NormalTok{, }\DecValTok{100}\NormalTok{)) }\OperatorTok{+}\StringTok{   }
\StringTok{                                         }\KeywordTok{scale_x_continuous}\NormalTok{(}\DataTypeTok{breaks =} \KeywordTok{seq}\NormalTok{(}\DecValTok{2005}\NormalTok{,}\DecValTok{2019}\NormalTok{, }\DecValTok{2}\NormalTok{))}\OperatorTok{+}
\StringTok{                                         }\KeywordTok{theme}\NormalTok{(}\DataTypeTok{legend.position =} \StringTok{"bottom"}\NormalTok{, }\DataTypeTok{legend.box =} \StringTok{"horizontal"}\NormalTok{)}\OperatorTok{+}
\StringTok{                                          }\KeywordTok{labs}\NormalTok{(}\DataTypeTok{title =} \StringTok{"Indicador de verdadeiro-positivo (C) por canal de atendimento"}\NormalTok{,}
                                              \DataTypeTok{caption =} \StringTok{"Período analisado: 07/05/2020 a 31/12/2019"}\NormalTok{, }
                                              \DataTypeTok{x =} \StringTok{"Ano"}\NormalTok{,}
                                              \DataTypeTok{y =} \StringTok{"C"}\NormalTok{,}
                                              \DataTypeTok{colour =} \StringTok{"Canal de aberura"}\NormalTok{) }
\end{Highlighting}
\end{Shaded}

\begin{verbatim}
## `summarise()` regrouping output by 'year(data_solic)', 'ocorr_solic' (override with `.groups` argument)
\end{verbatim}

\includegraphics{Corresp_files/figure-latex/unnamed-chunk-10-2.pdf}

\begin{Shaded}
\begin{Highlighting}[]
\NormalTok{sdata[sdata}\OperatorTok{$}\NormalTok{ocorr_solic }\OperatorTok{==}\StringTok{ "ALAGAMENTO"} \OperatorTok{&}\StringTok{ }\NormalTok{sdata}\OperatorTok{$}\NormalTok{desc_orig_solic }\OperatorTok\StringTok{ }\KeywordTok{c}\NormalTok{(}\StringTok{"199"}\NormalTok{, }\StringTok{"PESSOALMENTE"}\NormalTok{),] }\OperatorTok\StringTok{ }\KeywordTok{group_by}\NormalTok{(}\KeywordTok{year}\NormalTok{(data_solic), desc_orig_solic) }\OperatorTok\StringTok{ }
\StringTok{                                         }\KeywordTok{summarise}\NormalTok{(}\DataTypeTok{n =} \KeywordTok{mean}\NormalTok{(match)}\OperatorTok{*}\DecValTok{100}\NormalTok{) }\OperatorTok\StringTok{ }
\StringTok{                                         }\KeywordTok{ggplot}\NormalTok{() }\OperatorTok{+}\StringTok{ }\KeywordTok{geom_line}\NormalTok{(}\KeywordTok{aes}\NormalTok{(}\DataTypeTok{x =} \StringTok{`}\DataTypeTok{year(data_solic)}\StringTok{`}\NormalTok{, }\DataTypeTok{y =}\NormalTok{ n, }\DataTypeTok{colour =} \KeywordTok{as.factor}\NormalTok{(desc_orig_solic)), }\DataTypeTok{size =} \FloatTok{1.2}\NormalTok{) }\OperatorTok{+}\StringTok{                      }
\StringTok{                                          }\KeywordTok{coord_cartesian}\NormalTok{(}\DataTypeTok{xlim =}\KeywordTok{c}\NormalTok{(}\DecValTok{2005}\NormalTok{, }\DecValTok{2019}\NormalTok{), }\DataTypeTok{ylim =} \KeywordTok{c}\NormalTok{(}\DecValTok{0}\NormalTok{, }\DecValTok{100}\NormalTok{)) }\OperatorTok{+}\StringTok{   }
\StringTok{                                         }\KeywordTok{scale_x_continuous}\NormalTok{(}\DataTypeTok{breaks =} \KeywordTok{seq}\NormalTok{(}\DecValTok{2005}\NormalTok{,}\DecValTok{2019}\NormalTok{, }\DecValTok{2}\NormalTok{))}\OperatorTok{+}
\StringTok{                                         }\KeywordTok{theme}\NormalTok{(}\DataTypeTok{legend.position =} \StringTok{"bottom"}\NormalTok{, }\DataTypeTok{legend.box =} \StringTok{"horizontal"}\NormalTok{)}\OperatorTok{+}
\StringTok{                                          }\KeywordTok{labs}\NormalTok{(}\DataTypeTok{title =} \StringTok{"Indicador de verdadeiro-positivo (C) por canal de atendimento"}\NormalTok{,}
                                              \DataTypeTok{caption =} \StringTok{"Período analisado: 07/05/2020 a 31/12/2019"}\NormalTok{, }
                                              \DataTypeTok{x =} \StringTok{"Ano"}\NormalTok{,}
                                              \DataTypeTok{y =} \StringTok{"C"}\NormalTok{,}
                                              \DataTypeTok{colour =} \StringTok{"Canal de aberura"}\NormalTok{) }
\end{Highlighting}
\end{Shaded}

\begin{verbatim}
## `summarise()` regrouping output by 'year(data_solic)' (override with `.groups` argument)
\end{verbatim}

\includegraphics{Corresp_files/figure-latex/unnamed-chunk-10-3.pdf}

\end{document}
